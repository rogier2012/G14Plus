%%%%%%%%%%%%%%%%%%%%%%%%%%%%%%%%%%%%%%%%%
% Journal Article
% LaTeX Template
% Version 1.3 (9/9/13)
%
% This template has been downloaded from:
% http://www.LaTeXTemplates.com
%
% Slightly updated by  Marc Uetz
%
% Original author:
% Frits Wenneker (http://www.howtotex.com)
%
% License:
% CC BY-NC-SA 3.0 (http://creativecommons.org/licenses/by-nc-sa/3.0/)
%
%%%%%%%%%%%%%%%%%%%%%%%%%%%%%%%%%%%%%%%%%

%----------------------------------------------------------------------------------------
%	PACKAGES AND OTHER DOCUMENT CONFIGURATIONS
%----------------------------------------------------------------------------------------

\documentclass[twoside]{article}

\usepackage{lipsum} % Package to generate dummy text throughout this template

\usepackage{amsmath,amssymb,amsthm} % Mathematical Symbols, styles, etc

\newtheorem{theorem}{Theorem}[section]
\newtheorem{lemma}[theorem]{Lemma}
\newtheorem{proposition}[theorem]{Proposition}
\newtheorem{corollary}[theorem]{Corollary}

\usepackage[sc]{mathpazo} % Use the Palatino font
\usepackage[T1]{fontenc} % Use 8-bit encoding that has 256 glyphs
\linespread{1.05} % Line spacing - Palatino needs more space between lines
\usepackage{microtype} % Slightly tweak font spacing for aesthetics

\usepackage[hmarginratio=1:1,top=32mm,columnsep=20pt]{geometry} % Document margins
\usepackage{multicol} % Used for the two-column layout of the document
\usepackage[hang, small,labelfont=bf,up,textfont=it,up]{caption} % Custom captions under/above floats in tables or figures
\usepackage{booktabs} % Horizontal rules in tables
\usepackage{float} % Required for tables and figures in the multi-column environment - they need to be placed in specific locations with the [H] (e.g. \begin{table}[H])
\usepackage{hyperref} % For hyperlinks in the PDF

\usepackage{lettrine} % The lettrine is the first enlarged letter at the beginning of the text
\usepackage{paralist} % Used for the compactitem environment which makes bullet points with less space between them

\usepackage{abstract} % Allows abstract customization
\renewcommand{\abstractnamefont}{\normalfont\bfseries} % Set the "Abstract" text to bold
\renewcommand{\abstracttextfont}{\normalfont\small\itshape} % Set the abstract itself to small italic text

\usepackage{titlesec} % Allows customization of titles
%\renewcommand\thesection{\Roman{section}} % Roman numerals for the sections
%\renewcommand\thesubsection{\Roman{subsection}} % Roman numerals for subsections
\titleformat{\section}[block]{\large\scshape}{\thesection.}{1em}{} % Change the look of the section titles
\titleformat{\subsection}[block]{\large}{\thesubsection.}{1em}{} % Change the look of the section titles

\usepackage{fancyhdr} % Headers and footers
\pagestyle{fancy} % All pages have headers and footers
\fancyhead{} % Blank out the default header
\fancyfoot{} % Blank out the default footer
\fancyhead[C]{J.\ Smith, M.\ Koopmans, P.\ Blank: \shorttitle} % Custom header text
\fancyfoot[RO,LE]{\thepage} % Custom footer text

%----------------------------------------------------------------------------------------
%	TITLE SECTION
%----------------------------------------------------------------------------------------

\newcommand{\articletitle}{Fast partition refinement, a Python implementation}
\newcommand{\shorttitle}{Python fast partitioning}

\title{\vspace{-15mm}\fontsize{24pt}{10pt}\selectfont\textbf{\articletitle}} % Article title

\author{
\large
\textsc{Christiaan van den Bogaard, Dion Ikink, Fleur Seuren, and Rogier Monshouwer}\thanks{Thanks to our helpful mentor.}\\[2mm] % Your name
\normalsize University of Twente \\ % Your institution
\normalsize \href{mailto:d.t.r.ikink@student.utwente.nl}{d.t.r.ikink@student.utwente.nl}, 
\href{mailto:c.h.m.vandenbogaard@student.utwente.nl}{c.h.m.vandenbogaard@student.utwente.nl} \\
\normalsize\href{mailto:f.seuren@student.utwente.nl}{f.seuren@student.utwente.nl}, % Your email addresses
\href{mailto:r.monshouwer@student.utwente.nl}{r.monshouwer@student.utwente.nl}
}

\date{\today}

%----------------------------------------------------------------------------------------

\begin{document}

\thispagestyle{empty}
\maketitle % Insert title

%----------------------------------------------------------------------------------------
%	ABSTRACT
%----------------------------------------------------------------------------------------

\begin{abstract}

\noindent \lipsum[1] % Dummy abstract text

\end{abstract}

%----------------------------------------------------------------------------------------
%	ARTICLE CONTENTS
%----------------------------------------------------------------------------------------

\begin{multicols}{2} % Two-column layout throughout the main article text



%------------------------------------------------
\section{Introduction}

%\lettrine[nindent=0em,lines=3]{L} orem ipsum dolor sit amet, consectetur adipiscing elit.
\lipsum[2-3] % Dummy text

%------------------------------------------------

\section{Scope \& Objectives}

Describe \lipsum[1] % Dummy text

%------------------------------------------------
\section{Methods}
This results in the following, exhaustive list of possibilities. 
\begin{compactitem}
\item Donec dolor arcu, rutrum id molestie in, viverra sed diam
\item Curabitur feugiat
\item turpis sed auctor facilisis
\item arcu eros accumsan lorem, at posuere mi diam sit amet tortor
\item Fusce fermentum, mi sit amet euismod rutrum
\item sem lorem molestie diam, iaculis aliquet sapien tortor non nisi
\item Pellentesque bibendum pretium aliquet
\end{compactitem}
\lipsum[4] % Dummy text

%------------------------------------------------
\section{Results}

The results with this algorithm are given in Table~\ref{table:example}.
\begin{table}[H]
\caption{Example table}\label{table:example}
\centering
\begin{tabular}{llr}
\toprule
\multicolumn{2}{c}{Name} \\
\cmidrule(r){1-2}
First name & Last Name & Grade \\
\midrule
John & Doe & $7.5$ \\
Richard & Miles & $2$ \\
\bottomrule
\end{tabular}
\end{table}


\lipsum[5] % Dummy text

%------------------------------------------------
\subsection{Implementation}
\lipsum[1]
The following therefore holds for all
planar graphs $G$ with $n$ vertices, $m$ edges, and $r$ regions.

\begin{equation}
\label{eq:emc}
n-m+r=2
\end{equation}

%------------------------------------------------
\subsection{Computational Results}

The following conclusion follows directly from \eqref{eq:emc}.
\lipsum[6] % Dummy text
\begin{theorem}
Whenever two graphs $G$ and $G'$ are isomorphic, they share the same degree sequences.
\end{theorem}
\begin{proof}
Obvious.
\end{proof}
We immediately conclude the following.
\begin{corollary}
Whenever two planar graphs $G$ and $G'$ are isomorphic, their maximum degree vertices have the same degree. 
\end{corollary}
This observation first appeared in , and was later generalized to arbitrary graphs in .
In our implementation, we have therefore used the following algorithmic idea.
\begin{verbatim}
def gcd(a,b):
    if b > a:
        a,b = b,a
    r = a%b
    while r>0:
        a = b
        b = r
        r = a%b
    return b
\end{verbatim}
For more information on programming in {\tt python}, we refer to\cite{jaco} \cite{MR0403320}.

%------------------------------------------------
\section{Conclusions}


\lipsum[8] % Dummy text. 


%----------------------------------------------------------------------------------------
%	REFERENCE LIST
%----------------------------------------------------------------------------------------
 
% Bibliography - this is intentionally simple in this template
\bibliographystyle{plain}
\bibliography{referenties}

%----------------------------------------------------------------------------------------

\end{multicols}

\end{document}
